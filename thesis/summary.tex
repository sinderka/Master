% !TEX encoding = UTF-8 Unicode
%!TEX root = thesis.tex
% !TEX spellcheck = en-US
%%=========================================
\addcontentsline{toc}{section}{Summary and Conclusions}
\section*{Summary and Conclusions}
Krylov methods are projection methods that can transform big linear ODE's to a smaller linear ODE's with similar properties. 
Two such methods (the symplectic Lanczos method (\texttt{SLM}) and the Krylov projection method (\texttt{KPM})) are compared with each other on linear Hamiltonian differential equations. The behavior of global error, energy as a function of time is examined. Computation time for the different methods are also shown. The projection methods are also compared to each other on non autonomous linear Hamiltonian differential equation.
Energy preservation for the symplectic Lanczos method is proved under some assumptions, along with convergence for both projection methods. \texttt{SLM} has no advantage over \texttt{KPM} when considering error. On autonomous systems the computation times for \texttt{SLM} and \texttt{KPM} are similar, and both methods can be well utilized. On non autonomous systems \texttt{KPM} is faster than \texttt{SLM}. Since it has comparable error and energy preservation \texttt{KPM} might be considered a better choice in this case.


