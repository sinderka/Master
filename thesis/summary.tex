% !TEX encoding = UTF-8 Unicode
%!TEX root = thesis.tex
% !TEX spellcheck = en-US
%%=========================================
\addcontentsline{toc}{section}{Summary and Conclusions}
\section*{Summary and Conclusions}
Two projection methods (the symplectic Lanczos method and the Krylov projection method) are compared to each other on linear Hamiltonian systems. Energy, error and computation time is the primary concern. Energy preservation for the symplectic Lanczos method is proved and shown under some assumptions, along with convergence for both projection methods. \\
The projection methods are also compered to each other on non-autonomous linear Hamiltonian systems.

%Solving partial differential equations with finite difference methods often requires performing operations on huge linear systems. The Krylov projection method allows the user to choose the size of the linear system by making the method iterative. The Krylov projection method was tested on the heat equation against other methods in convergence, memory demand and computation time. \\

%The results shows that the Krylov projection method can reduce memory demand, and computation time if several processing units are used or some assumptions are met. Convergence for the Krylov projection method was found to be indistinguishable from other methods .

%No difference regarding convergence was found.