%%%%%%%%%%%%%%%%%%%%%%%%%%%%%%%%%%%%%%%%%%%%%%%%%%%%%%%%%%%%%%%%%%%%%%%%%%%%%%%%%%%%%%%%%%%%%%%%%%%%%%%%%%%%%%%%%%%%%%
\chapter{Introduction} %%%%%%%%%%%%%%%%%%%%%%%%%%%%%%%%%%%%%%%%%%%%%%%%%%%%%%%%%%%%%%%%%%%%%%%%%%%%%%%%%%%%%%%%%%%%%%%
%%%%%%%%%%%%%%%%%%%%%%%%%%%%%%%%%%%%%%%%%%%%%%%%%%%%%%%%%%%%%%%%%%%%%%%%%%%%%%%%%%%%%%%%%%%%%%%%%%%%%%%%%%%%%%%%%%%%%%
A matrix $A$ is said to be Hamiltonian if  \cite{Hamiltonian}
\begin{equation}
(JA)^{\top} = J A. 
\end{equation}
where
\begin{equation}
J_j = 
\begin{bmatrix}
0&I_j\\-I_j&0
\end{bmatrix}.
\end{equation}
and $I_j$ is the $j \times j$ identity matrix.
Hamiltonian systems on the form of equation \eqref{eqn:PDE} is found in several branches of physics and mathematics, eg. planetary movement, particle physics and control theory.
\begin{equation} 
\begin{aligned}
\dot{u}(t) &= A u(t)\\
u(0)&= u_0.
\end{aligned}
\label{eqn:PDE}
\end{equation}
The matrix $A$ is in theses cases large sparse and Hamiltonian. Acquiring good approximations can be computationally costly. A projection method that exploits these properties is the symplectic Lanczos method (\texttt{SLM}). \texttt{SLM} projects a large matrix $A$ onto a smaller dimensional Hamiltonian problem, making further calculations less computationally demanding. \texttt{SLM} has in \citep{SLM1}, \citep{SLM2} and \citep{SLM3} been used to find eigenvalues of large Hamiltonian matrices. \texttt{SLM} preserves the Hamiltonian structure of the matrix, which allows the energy of the system to be preserved.\\
Arnoldi's algorithm, together with the Krylov projection method (KPM) is another much used projection method. \texttt{KPM} has no structure preserving properties, which makes it faster at the cost of energy preservation. \\
This text will look at how these two projection methods compare to alternatives without projecting, in error approximation, energy preservation, and computation time. \\
Equation \eqref{eqn:PDE} has the well known analytical solution 
\begin{equation*}
u(t) = \text{exp}(At)u_0.
\end{equation*}
Taking the matrix exponential is a very costly operation, but can be a great way to exploit the smaller matrices produced by the projection methods. This will also be looked at closely.\\
In addition to the case with Hamiltonian systems and constant energy, the projection methods will be tested against each other when the energy is non constant.
Matlab notation will be used where applicable.\\
%\begin{itemize}
%\item Løsningen på ligning 1 må skrives et sted (introduksjon)
%\item er SLM uten restart energi bevarende
%\item er SLM med restart energi bevarende
%\item Hvor raskt øker feilen for SLM med og uten restart.
%\item Skriv mer i introduksjonskap
%\item Skrive om applikasjoner i introduksjonen
%\item Det må overalt stå om ting er vektor eller skalar funksjon! (intro kap)
%\item restarting må nevnes
%\item 
%\end{itemize}
