%%%%%%%%%%%%%%%%%%%%%%%%%%%%%%%%%%%%%%%%%%%%%%%%%%%%%%%%%%%%%%%%%%%%%%%%%%%%%%%%%%%%%%%%%%%%%%%%%%%%%%%%%%%%%%%%%%%%%%
\chapter{Introduction} %%%%%%%%%%%%%%%%%%%%%%%%%%%%%%%%%%%%%%%%%%%%%%%%%%%%%%%%%%%%%%%%%%%%%%%%%%%%%%%%%%%%%%%%%%%%%%%
%%%%%%%%%%%%%%%%%%%%%%%%%%%%%%%%%%%%%%%%%%%%%%%%%%%%%%%%%%%%%%%%%%%%%%%%%%%%%%%%%%%%%%%%%%%%%%%%%%%%%%%%%%%%%%%%%%%%%%
A matrix $A$ is said to be Hamiltonian if  \cite{Hamiltonian}
\begin{equation*}
(J_jA)^{\top} = J_j A,
\end{equation*}
where
\begin{equation*}
J_j = 
\begin{bmatrix}
0&I_j\\-I_j&0
\end{bmatrix},
\end{equation*}
and $I_j$ is the $j \times j$ identity matrix.
Hamiltonian systems on the form 
\begin{equation} 
\begin{aligned}
\dot{u}(t) &= A u(t)\\
u(0)&= u_0.
\end{aligned}
\label{eqn:PDE}
\end{equation} is found in several branches of physics and mathematics, eg. planetary movement, particle physics and control theory.
The matrix $A$ is in theses cases large sparse and Hamiltonian. Acquiring good approximations can be computationally costly. A projection method that exploits the property of these matrices is the symplectic Lanczos method (\texttt{SLM}). \texttt{SLM} projects a large matrix $A$ onto a smaller dimensional Hamiltonian problem, making further calculations less computationally demanding. This property has made \texttt{SLM} a popular choice when finding eigenvalues of large Hamiltonian matrices, see eg. \citep{SLM1}, \citep{SLM2} and \citep{SLM3}. Since \texttt{SLM} preserves the Hamiltonian structure of the matrix, energy of the system \eqref{eqn:PDE} is preserved.\\
Arnoldi's algorithm, together with the Krylov projection method (KPM) is another much used projection method. \texttt{KPM} has no structure preserving properties, which makes it faster at the cost of energy preservation. \\
This text will look at how these two projection methods compare to alternatives without projecting: in error approximation, energy preservation, and computation time. \\
Equation \eqref{eqn:PDE} has the well known analytical solution 
\begin{equation*}
u(t) = \text{exp}(At)u_0.
\end{equation*}
Taking the matrix exponential is a very costly operation, but can be a great way to exploit the smaller matrices produced by the projection methods. This will also be looked at closely.\\
In addition to the case with Hamiltonian systems and constant energy, the projection methods will be tested against each other when the energy is nonconstant as in equation \eqref{eqn:PDE1}.
\begin{equation}
\begin{aligned}
\dot{u}(t) &= A u(t) + p f(t)\\
u(0)&= u_0,
\end{aligned}
\label{eqn:PDE1}
\end{equation}
where $p$ is a vector, and $f(t)$ is a scalar time dependent function.
Matlab notation will be used where applicable.\\

