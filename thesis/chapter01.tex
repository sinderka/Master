%%%%%%%%%%%%%%%%%%%%%%%%%%%%%%%%%%%%%%%%%%%%%%%%%%%%%%%%%%%%%%%%%%%%%%%%%%%%%%%%%%%%%%%%%%%%%%%%%%%%%%%%%%%%%%%%%%%%%%
\chapter{Introduction}
%%%%%%%%%%%%%%%%%%%%%%%%%%%%%%%%%%%%%%%%%%%%%%%%%%%%%%%%%%%%%%%%%%%%%%%%%%%%%%%%%%%%%%%%%%%%%%%%%%%%%%%%%%%%%%%%%%%%%%

The equation 
\begin{equation} 
\begin{aligned}
\dot{u}(t) &= A u(t) + F(t) = g(t) \\
u(0)&= u_0
\end{aligned}
\label{eqn:PDE}
\end{equation}
often makes an appearance when solving partial differential equations with numerical methods. The author has earlier observed how the heat equation, discretized with finite difference methods to be on the form of equation \eqref{eqn:PDE} can be solved with the use of the Krylov projection method(KPM) \cite{min}, which uses Arnoldi's algorithm(Arnoldi) as orthogonalisation method. This note will continue on the same track, but with more focus on Hamiltonian discretizations of wave equations, random problems, and energy preservation. It will also feature a comparison between symplectic Lanzcos method(SLM) \cite{SLM} and Arnoldi. SLM is a projection technique that only works on Hamiltonian matrices. Due to this, SLM (claims to) preserve energy better than Arnoldi. This note will also compare time consumption and error for the different methods, but will not contain much theoretical derivation. For this I recommend reading \cite{elena}, \cite{min}, \citep{SLM}, \cite{SLMO}, and \cite{luli}. \\



