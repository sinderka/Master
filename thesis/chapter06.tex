%%%%%%%%%%%%%%%%%%%%%%%%%%%%%%%%%%%%%%%%%%%%%%%%%%%%%%%%%%%%%%%%%%%%%%%%%%%%%%%%%%%%%%%%%%%%%%%%%%%%%%%%%%%%%%%%%%%%%%
\chapter{Results for test problems with varying energy}%%%%%%%%%%%%%%%%%%%%%%%%%%%%%%%%%%%%%
\label{sec:varyener} %%%%%%%%%%%%%%%%%%%%%%%%%%%%%%%%%%%%%%%%%%%%%%%%%%%%%%%%%%%%%%%%%%%%%%%%%%%%%%%%%%%%%%%%%%%%%%
This chapter will look at many of the same elements that was discussed in section \ref{sec:constres}. Some results are similar, in this case results may not be shown here. This chapter will try to find out if there is any reason to use \texttt{SLM} instead of \texttt{KPM} on non autonomous Hamiltonian systems. \\
The exact solvers does not work when the energy is varying, thus there will be no discussion about that in this chapter.
\section{Convergence} %%%%%%%%%%%%%%%%%%%%%%%%%%%%%%%%%%%%%%%%%%%%%%%%%%%%%%%%%%%%%%%%%%%
\begin{figure}[H]
        \centering
        
		
		\begin{subfigure}[b]{0.45\textwidth}
                \includegraphics[width=\textwidth]{../MATLAB/fig/varconv11r.jpg}
                \caption{ Convergence for \texttt{wave} with trapezoidal rule without restart. }
                \label{fig:varconv11r}
        \end{subfigure}%
        ~
        \begin{subfigure}[b]{0.45\textwidth}
                \includegraphics[width=\textwidth]{../MATLAB/fig/varconv13r.jpg}
                \caption{ Convergence for \texttt{wave} with midpoint rule without restart. }
                \label{fig:varconv13r}
        \end{subfigure}
        \begin{subfigure}[b]{0.45\textwidth}
                \includegraphics[width=\textwidth]{../MATLAB/fig/varconv11.jpg}
                \caption{ Convergence for \texttt{wave} with trapezoidal rule with restart. }
                \label{fig:varconv11}
        \end{subfigure}%
        ~
        \begin{subfigure}[b]{0.45\textwidth}
                \includegraphics[width=\textwidth]{../MATLAB/fig/varconv13.jpg}
                \caption{ Convergence for \texttt{wave} with midpoint rule rule with restart. }
                \label{fig:varconv13}
        \end{subfigure}

        \caption{ Convergence plot with different integrators, simulated over 1 second. }
        \label{fig:varconv}
\end{figure}

Midpoint rule performs better than trapezoidal rule. It is interesting to see that without restart the methods does not converge with $n = 20$, while they converged with $n = 2$ when the energy was constant. Clearly convergence is a lot harder to achieve in this case. \\
Only midpoint rule will be used further in this chapter.\\

\section{Convergence with $\iota$} %%%%%%%%%%%%%%%%%%%%%%%%%%%%%%%%%%%%%%%%%%%%%%%%%%%%%%%

\begin{figure}[H]
        \centering
        \begin{subfigure}[b]{0.45\textwidth}
                \includegraphics[width=\textwidth]{../MATLAB/fig/varyEnergy.jpg}
                \caption{ Energy for \texttt{wave}. }
                \label{fig:varyEnergy}
        \end{subfigure}%
        ~
        \begin{subfigure}[b]{0.45\textwidth}
                \includegraphics[width=\textwidth]{../MATLAB/fig/varyEnergyw.jpg}
                \caption{ Energy for \texttt{semirandom}. }
                \label{fig:varyEnergyw}
        \end{subfigure}
		
		\begin{subfigure}[b]{0.45\textwidth}
                \includegraphics[width=\textwidth]{../MATLAB/fig/varyError.jpg}
                \caption{ Error for \texttt{wave}. }
                \label{fig:varyError}
        \end{subfigure}%
        ~
        \begin{subfigure}[b]{0.45\textwidth}
                \includegraphics[width=\textwidth]{../MATLAB/fig/varyErrorw.jpg}
                \caption{ Error for \texttt{semirandom}. }
                \label{fig:varyErrorw}
        \end{subfigure}
        
        		\begin{subfigure}[b]{0.45\textwidth}
                \includegraphics[width=\textwidth]{../MATLAB/fig/varyIter.jpg}
                \caption{ Number of restarts for \texttt{wave} }
                \label{fig:varyIter}
        \end{subfigure}%
        ~
        \begin{subfigure}[b]{0.45\textwidth}
                \includegraphics[width=\textwidth]{../MATLAB/fig/varyIterw.jpg}
                \caption{ Number of restarts for \texttt{semirandom}. }
                \label{fig:varyIterw}
        \end{subfigure}
        \caption{ These figures show how restarting can improve the solution. The pictures on the left is for \texttt{wave} and one the right is for \texttt{semirandom}. This plot considers 100 seconds, with  $n=m=20$, $k = 2000$, $m = 20$, and midpoint rule. }
        \label{fig:variota}
\end{figure}
These figures look very similar to the figures in section \ref{fig:lcompare}, one important difference is that the energy for \texttt{SLM} no longer starts at $1e-15$. The difference between \texttt{SLM} and \texttt{KPM} is definitely smaller in this case.


\section{Energy and error} %%%%%%%%%%%%%%%%%%%%%%%%%%%%%%%%%%%%%%%%%%%%%%%%%%%%%%%%%%%%%
The dependance between $n$ is the same as in section \ref{sec:resultat}, thus $n$ will be kept at the same value it was. The results in this section is very similar to the results in section \ref{sec:resultconsterergy}. Results not shown are to similar to show.
\subsection{Without restart} %%%%%%%%%%%%%%%%%%%%%%%%%%%%%%%%%%%%%%%%%%%%%%%%%%%%%%%%%%%%%
\begin{figure}[H]
        \centering
        \begin{subfigure}[b]{0.45\textwidth}
                \includegraphics[width=\textwidth]{../MATLAB/fig/vlongtime2err.jpg}
                \caption{ Error. }
                \label{fig:vlongtime2err}
        \end{subfigure}
        ~
        \begin{subfigure}[b]{0.45\textwidth}
                \includegraphics[width=\textwidth]{../MATLAB/fig/vlongtime2ene.jpg}
                \caption{ Energy. }
                \label{fig:vlongtime8err}
        \end{subfigure}
        \caption{ The change in error and energy as a function of time. $m = 20$, restart is not enabled. }
        \label{fig:vSLMenergyerror0}
\end{figure}
\noindent Figure \ref{fig:vSLMenergyerror0} is very similar to Figure \ref{fig:SLMenergyerror0}. The only difference is the more powerful divergence occurring at the last points. \texttt{SLM} manages to maintain a good approximation of the energy throughout the entire time domain. This shows that \texttt{SLM} may still be useful in this setting.
\subsection{With restart} %%%%%%%%%%%%%%%%%%%%%%%%%%%%%%%%%%%%%%%%%%%%%%%%%%%%%%%%%%%%%%%
\begin{figure}[H]
        \centering
        \begin{subfigure}[b]{0.45\textwidth}
                \includegraphics[width=\textwidth]{../MATLAB/fig/vlongtime2rerr.jpg}
                \caption{ Error. }
                \label{fig:vlongtime2rerr}
        \end{subfigure}
        ~
        \begin{subfigure}[b]{0.45\textwidth}
                \includegraphics[width=\textwidth]{../MATLAB/fig/vlongtime2rene.jpg}
                \caption{ Energy. }
                \label{fig:vlongtime8rerr}
        \end{subfigure}
     
        
        \begin{subfigure}[b]{0.45\textwidth}
                \includegraphics[width=\textwidth]{../MATLAB/fig/vlongtime2rite.jpg}
                \caption{ Number of restarts. }
                \label{fig:vlongtime2rene}
        \end{subfigure}
        \caption{ The top pictures shows error and energy, while the bottom picture shows the number of iterations. $m = 20$, restart is enabled with $\iota = 1e-6$. }
        \label{fig:vSLMenergyerror1}
\end{figure}
The restart makes \texttt{SLM} and \texttt{KPM} behave very similar. No divergence occurs on this small time interval. Restarting is a better idea when the energy is varying than when the energy is constant. These figures are very comparable to Figure \ref{fig:SLMenergyerror1}. Even thoug it is not shown here the methods diverges on time domains 
\subsection{Windowing}%%%%%%%%%%%%%%%%%%%%%%%%%%%%%%%%%%%%%%%%%%%%%%%%%%%%%%%%%%%%%%%%%%%%
\begin{figure}[H]
        \centering
        \begin{subfigure}[b]{0.45\textwidth}
                \includegraphics[width=\textwidth]{../MATLAB/fig/lversuskerror0.jpg}
                \caption{ Error with restart. }
                \label{fig:lversuskerror0}
        \end{subfigure}
		~
		\begin{subfigure}[b]{0.45\textwidth}
                \includegraphics[width=\textwidth]{../MATLAB/fig/lversuskenergy0.jpg}
                \caption{ Energy with restart. }
                \label{fig:lversuskenergy0}
        \end{subfigure}
                \caption{ Windowing with $m = 20$, $k= 20$. The restart does not change the result.}
        \label{fig:lversuskenergy}
\end{figure}
\noindent Figure \ref{fig:lversuskenergy} shows that windowing does not work with varying energy, thus a very promising solution strategy from the previous chapter has a limiting factor.\\


\section{Computation time} %%%%%%%%%%%%%%%%%%%%%%%%%%%%%%%%%%%%%%%%%%%%%%%%%%%%%%%%%%%%%
Computation times are fairly different due to extra difficulties with convergence.
\subsection{Without restart}
\begin{figure}[H]
        \centering
        \begin{subfigure}[b]{0.45\textwidth}
                \includegraphics[width=\textwidth]{../MATLAB/fig/ltimem.jpg}
                \caption{ Computation time as a function of $m$. }
                \label{fig:ltimem}
        \end{subfigure}
        ~
        \begin{subfigure}[b]{0.45\textwidth}
                \includegraphics[width=\textwidth]{../MATLAB/fig/ltimek.jpg}
                \caption{ Computation time as a function of $k$. }
                \label{fig:ltimek}
        \end{subfigure}
        \caption{ A figure of the computation times without restart. $n = 200$, $T_s = 100$, $k = 2000$, and $m = 20$ unless stated. }
        \label{fig:ltime0}
\end{figure}
The difference between \texttt{DM} and the projection methods are a lot smaller in this case. At the last point on figure \ref{fig:ltime0}, \texttt{DM} and \texttt{KPM} are intersecting. \texttt{KPM} becomes faster than \texttt{DM} after this, but \texttt{SLM} does not overtake \texttt{DM}. \\
These results are a little surprising since the numbers are the same as in section \ref{sec:cruntime}, and there is no restarting. 

\subsection{With restart}

\begin{figure}[H]
        \centering
        \begin{subfigure}[b]{0.45\textwidth}
                \includegraphics[width=\textwidth]{../MATLAB/fig/ltimemr.jpg}
                \caption{ Computation time as a function of $m$. }
                \label{fig:ltimemr}
        \end{subfigure}
        ~
        \begin{subfigure}[b]{0.45\textwidth}
                \includegraphics[width=\textwidth]{../MATLAB/fig/ltimekr.jpg}
                \caption{ Computation time as a function of $k$. }
                \label{fig:ltimekr}
        \end{subfigure}
        
        \begin{subfigure}[b]{0.45\textwidth}
                \includegraphics[width=\textwidth]{../MATLAB/fig/ltimekr1.jpg}
                \caption{ Computation time as a function of $n$. }
                \label{fig:ltimekr1}
        \end{subfigure}
        \caption{ A figure of the computation times with restart. $n = 20$, $T_s = 100$, $k = 2000$, and $m = 20$ unless stated. }
        \label{fig:ltime1}
\end{figure}
Figure \ref{fig:ltime1} shows that it can be a good idea to use \texttt{KPM} when $m$ is large and $k$ is small. The faster increase with $k$ for \texttt{KPM} comes from extra number of restarts that \texttt{KPM} needs to ensure convergence. The relation between $n$ and computation time shows that \texttt{KPM} has the possibility if being alot \cite{alot} faster than \texttt{DM}, \texttt{SLM} does not have this possibility. This makes \texttt{SLM} consistently slower than \texttt{DM}. \\
%It is never wise to use \texttt{SLM} in this case.
