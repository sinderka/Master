%%%%%%%%%%%%%%%%%%%%%%%%%%%%%%%%%%%%%%%%%%%%%%%%%%%%%%%%%%%%%%%%%%%%%%%%%%%%%%%%%%%%%%%%%%%%%%%%%%%%%%%%%%%%%%%%%%%%%%
\chapter{Results for test problems with varying energy}%%%%%%%%%%%%%%%%%%%%%%%%%%%%%%%%%%%%%
\label{sec:varyener} %%%%%%%%%%%%%%%%%%%%%%%%%%%%%%%%%%%%%%%%%%%%%%%%%%%%%%%%%%%%%%%%%%%%%%%%%%%%%%%%%%%%%%%%%%%%%%
In this chapter we will investigate the performance of \texttt{SLM} and \texttt{KPM} when applied to non autonomous Hamiltonian systems of type \eqref{eqn:PDE1}. The error and energy as a function of time $T_s$, together with computation time as a function of $m$ and $k$, is shown. $k$ is the number of steps in time, and $m$ is used to calculate the size of the matrix $\hat{m}$, given by the following relation $\hat{m} = 2(m-2)^2$. We also make a numerical study to find out what the best strategy for choosing the restart variable $n$ is.\\%How the restart variable $n$ can be chosen is also explored.\\

\noindent The exact solver in Table \ref{tab:intcorrect} does not work in this case due to the presence of the time dependent source term in equation \eqref{eqn:PDE1}. When \texttt{semirandom} is used, the error and energy will be measured as the difference between \texttt{DM} and the Krylov method used. This will be marked with $\text{error}^{\text{comp}}$ and $\text{energy}^{\text{comp}}$. This makes the error seem a lot smaller than it really is. 
Another approach that could be used is to apply the midpoint rule with a very small step-size, this has not been considered here.  \\

\noindent Since it is interesting to compare the results in this chapter to the previous chapter, all values will remain the same, meaning: $m = 8$, $n = 8$ with restart, $n = 40$ without restart, and $k = 20$ per second, unless stated otherwise. The tolerance $\iota = 1e-6$ is used on all figures with restart. 
\section{Convergence with $m$ and $k$} %%%%%%%%%%%%%%%%%%%%%%%%%%%%%%%%%%%%%%%%%%%%%%%%%%%%%%%%%%%%%%%%%%%
\label{sec:vconv}
\begin{figure}[H]
        \centering
		\begin{subfigure}[b]{0.3\textwidth}
                \includegraphics[width=\textwidth]{../MATLAB/fig/varconv11r.jpg}
                \caption{ Convergence for \texttt{wave} with trapezoidal rule without restart. }
                \label{fig:varconv11r}
        \end{subfigure}%
        ~
        \begin{subfigure}[b]{0.3\textwidth}
                \includegraphics[width=\textwidth]{../MATLAB/fig/varconv13r.jpg}
                \caption{ Convergence for \texttt{wave} with midpoint rule without restart.\\ }
                \label{fig:varconv13r}
        \end{subfigure}
        \begin{subfigure}[b]{0.3\textwidth}
                \includegraphics[width=\textwidth]{../MATLAB/fig/varconv11.jpg}
                \caption{ Convergence for \texttt{wave} with trapezoidal rule and restart. }
                \label{fig:varconv11}
        \end{subfigure}%
        ~
        \begin{subfigure}[b]{0.3\textwidth}
                \includegraphics[width=\textwidth]{../MATLAB/fig/varconv13.jpg}
                \caption{ Convergence for \texttt{wave} with midpoint rule and restart. }
                \label{fig:varconv13}
        \end{subfigure}

        \caption{ Convergence plot with different integrators, simulated over 1 second. The top pictures are without restart, the bottom pictures are with restart and $\iota = 1e-6$. }
        \label{fig:varconv}
\end{figure}

Figure \ref{fig:varconv} shows that without restart, the methods does not converge with $n = 20$, while they converged with $n = 2$ with constant energy (see Figure \ref{fig:intconv}). Clearly convergence is a lot harder to achieve in this case, though the restart can effectively be used to ensure this. With restart, all methods converge quadratically. The midpoint rule performs better than the trapezoidal rule, therefore only the midpoint rule will be used further in this chapter.\\

\section{Convergence with $\iota$} %%%%%%%%%%%%%%%%%%%%%%%%%%%%%%%%%%%%%%%%%%%%%%%%%%%%%%%

\begin{figure}[H]
        \centering
        \begin{subfigure}[b]{0.3\textwidth}
                \includegraphics[width=\textwidth]{../MATLAB/fig/varyEnergy.jpg}
                \caption{ Energy error for \texttt{wave}. }
                \label{fig:varyEnergy}
        \end{subfigure}%
        ~
		\begin{subfigure}[b]{0.3\textwidth}
                \includegraphics[width=\textwidth]{../MATLAB/fig/varyError.jpg}
                \caption{ Error for \texttt{wave}. }
                \label{fig:varyError}
        \end{subfigure}%
        ~
        \begin{subfigure}[b]{0.3\textwidth}
                \includegraphics[width=\textwidth]{../MATLAB/fig/varyIter.jpg}
                \caption{ Number of restarts for \texttt{wave} }
                \label{fig:varyIter}
        \end{subfigure}%
        
        \begin{subfigure}[b]{0.3\textwidth}
                \includegraphics[width=\textwidth]{../MATLAB/fig/varyEnergyw.jpg}
                \caption{ Energy error for \texttt{semirandom}. }
                \label{fig:varyEnergyw}
        \end{subfigure}
		~
        \begin{subfigure}[b]{0.3\textwidth}
                \includegraphics[width=\textwidth]{../MATLAB/fig/varyErrorw.jpg}
                \caption{ Error for \texttt{semirandom}. \\.}
                \label{fig:varyErrorw}
        \end{subfigure}
        ~
        \begin{subfigure}[b]{0.3\textwidth}
                \includegraphics[width=\textwidth]{../MATLAB/fig/varyIterw.jpg}
                \caption{ Number of restarts for \texttt{semirandom}. }
                \label{fig:varyIterw}
        \end{subfigure}
        \caption{ These figures show how restarting changes energy and error. The pictures on the top are for \texttt{wave} and the pictures on the bottom are for \texttt{semirandom}. The time integration is over a time interval of 1 second. }
        \label{fig:variota}
\end{figure}
Figure \ref{fig:variota} looks very similar to Figure \ref{fig:lcompare}, one important difference is that the energy for \texttt{SLM} no longer starts at $1e-15$, suggesting that \texttt{SLM} cannot estimate the energy precisely, without decreasing the error. The two Krylov methods behave very similarly. \\

\noindent \texttt{wave} has no need for the restart, while \texttt{semirandom} can have a great decrease in error and energy. In the interest of showing how the restart can best be utilized, only \texttt{semirandom} is used in the rest of this chapter.

\section{How to choose the restart variable $n$} \label{sec:resultatv}

\begin{figure}[H]
        \centering
        \begin{subfigure}[b]{0.30\textwidth}
                \includegraphics[width=\textwidth]{../MATLAB/fig/lnerrorwA.jpg}
                \caption{ Error for \texttt{KPM} without restart, for different $m$.\\. }
                \label{fig:lnerrorwA}
        \end{subfigure}
        ~
        \begin{subfigure}[b]{0.30\textwidth}
                \includegraphics[width=\textwidth]{../MATLAB/fig/lreserrA.jpg}
                \caption{ Error for \texttt{KPM} with restart, for different $m$.\\ }
                \label{fig:lreserrA}
        \end{subfigure}
        ~
        \begin{subfigure}[b]{0.30\textwidth}
                \includegraphics[width=\textwidth]{../MATLAB/fig/lterrorwA.jpg}
                \caption{ Error for \texttt{KPM} without restart, for different $T_s$.\\. }
                \label{fig:lterrorwA}
        \end{subfigure}

        
        \begin{subfigure}[b]{0.30\textwidth}
                \includegraphics[width=\textwidth]{../MATLAB/fig/lnerrorwS.jpg}
                \caption{ Error for \texttt{SLM} without restart, for different $m$.\\. }
                \label{fig:lnerrorwS}
        \end{subfigure}
		~
        \begin{subfigure}[b]{0.30\textwidth}
                \includegraphics[width=\textwidth]{../MATLAB/fig/lreserrS.jpg}
                \caption{ Error for \texttt{SLM} with restart, for different $m$.\\ }
                \label{fig:lreserrS}
        \end{subfigure}
        ~
        \begin{subfigure}[b]{0.30\textwidth}
                \includegraphics[width=\textwidth]{../MATLAB/fig/lterrorwS.jpg}
                \caption{ Error for \texttt{SLM} without restart, for different $T_s$.\\. }
                \label{fig:lterrorwS}
        \end{subfigure}
        
        
        \caption{ The pictures show which $n$ gives convergence for different $m$ and $T_s$. The size of the interval of integration is 10 seconds. }
        \label{fig:lerror}
\end{figure}
Figure \ref{fig:lerror} shows that $n$ depends linearly on both $m$ and $T_s$. This explains why the divergence in Section \ref{sec:vconv} occurs: $m$ becomes too large without $n$ increasing. If $n$ is too small, restarting will make the error increase. \\

\noindent The values chosen based on this picture will remain $n = 8$ when restart is not enabled, and $n = 40$ when restart is enabled. The reason for the large $n$ when restart is not enabled is that $T_s = 100$ is considered later.

\section{Energy and error as a function of time } %%%%%%%%%%%%%%%%%%%%%%%%%%%%%%%%%%%%%%%%%%%%%%%%%%%%%%%%%%%%%
This section will show how error and energy changes as a function of time. 

\subsection{Energy and error without windowing} %%%%%%%%%%%%%%%%%%%%%%%%%%%%%%%%%%%%%%%%%%%%%%%%%%%%%%%%%%%%%
\begin{figure}[H]
        \centering
        \begin{subfigure}[b]{0.3\textwidth}
                \includegraphics[width=\textwidth]{../MATLAB/fig/vlongtime2err.jpg}
                \caption{ Error without restart.\\. }
                \label{fig:vlongtime2err}
        \end{subfigure}
        ~
        \begin{subfigure}[b]{0.3\textwidth}
                \includegraphics[width=\textwidth]{../MATLAB/fig/vlongtime2ene.jpg}
                \caption{ Energy error without restart. }
                \label{fig:vlongtime8err}
        \end{subfigure}

        \begin{subfigure}[b]{0.3\textwidth}
                \includegraphics[width=\textwidth]{../MATLAB/fig/vlongtime2rerr.jpg}
                \caption{ Error with restart. }
                \label{fig:vlongtime2rerr}
        \end{subfigure}
        ~
        \begin{subfigure}[b]{0.3\textwidth}
                \includegraphics[width=\textwidth]{../MATLAB/fig/vlongtime2rene.jpg}
                \caption{ Energy error with restart. }
                \label{fig:vlongtime8rerr}
        \end{subfigure}
     	~
        \begin{subfigure}[b]{0.3\textwidth}
                \includegraphics[width=\textwidth]{../MATLAB/fig/vlongtime2rite.jpg}
                \caption{ Number of restarts. }
                \label{fig:vlongtime2rene}
        \end{subfigure}        
        
        \caption{ The pictures show the change in the error and energy as a function of time.  Restart is not enabled for the top pictures, restart is enabled for the bottom pictures. }
        \label{fig:vSLMenergyerror1}
\end{figure}
Figure \ref{fig:vSLMenergyerror1} shows that \texttt{SLM} preserves the energy a little longer than \texttt{KPM} when restart is not enabled. Apart from that, the methods are almost identical. These pictures look very similar to Figure \ref{fig:idea0}, this is because we only look at the error the Krylov methods create, and not the error made by the time integration method. The restart gives a much better estimate of the error and energy throughout the time domain, than when restart is not enabled. \\

\noindent Figure \ref{fig:vSLMenergyerror0} shows the divergence occurring on long time domains for the Krylov methods. Figure \ref{fig:longeriter} shows the linear increase in the number of restarts.

\begin{figure}[H]
        \centering
        \begin{subfigure}[b]{0.3\textwidth}
                \includegraphics[width=\textwidth]{../MATLAB/fig/longererr.jpg}
                \caption{ Error without restart.\\. }
                \label{fig:longererr}
        \end{subfigure}
        ~
        \begin{subfigure}[b]{0.3\textwidth}
                \includegraphics[width=\textwidth]{../MATLAB/fig/longerene.jpg}
                \caption{ Energy error without restart. }
                \label{fig:longerene}
        \end{subfigure}

        \begin{subfigure}[b]{0.3\textwidth}
                \includegraphics[width=\textwidth]{../MATLAB/fig/longererrr.jpg}
                \caption{ Error with restart. }
                \label{fig:longererrr}
        \end{subfigure}
        ~
        \begin{subfigure}[b]{0.3\textwidth}
                \includegraphics[width=\textwidth]{../MATLAB/fig/longerener.jpg}
                \caption{ Energy error with restart. }
                \label{fig:longerener}
        \end{subfigure}
     	~
        \begin{subfigure}[b]{0.3\textwidth}
                \includegraphics[width=\textwidth]{../MATLAB/fig/longeriter.jpg}
                \caption{ Number of restarts. }
                \label{fig:longeriter}
        \end{subfigure}        
        
        \caption{ The pictures show the change in error and energy over a long time domain. Restart is not enabled for the top pictures, restart is enabled for the bottom pictures. }
        \label{fig:vSLMenergyerror0}
\end{figure}

\subsection{Energy and error with windowing}%%%%%%%%%%%%%%%%%%%%%%%%%%%%%%%%%%%%%%%%%%%%%%%%%%%%%%%%%%%%%%%%%%%%
\begin{figure}[H]
        \centering
        \begin{subfigure}[b]{0.3\textwidth}
                \includegraphics[width=\textwidth]{../MATLAB/fig/lversuskerror0.jpg}
                \caption{ Error without restart.\\. }
                \label{fig:lversuskerror0}
        \end{subfigure}
		~
		\begin{subfigure}[b]{0.3\textwidth}
                \includegraphics[width=\textwidth]{../MATLAB/fig/lversuskenergy0.jpg}
                \caption{ Energy error without restart. }
                \label{fig:lversuskenergy0}
        \end{subfigure}
        
		\begin{subfigure}[b]{0.3\textwidth}
                \includegraphics[width=\textwidth]{../MATLAB/fig/lversuskerror0r.jpg}
                \caption{ Error with restart. }
                \label{fig:lversuskerror0r}
        \end{subfigure}
		~
		\begin{subfigure}[b]{0.3\textwidth}
                \includegraphics[width=\textwidth]{../MATLAB/fig/lversuskenergy0r.jpg}
                \caption{ Energy error with restart. }
                \label{fig:lversuskenergy0r}
        \end{subfigure}
                \caption{ The change in error and energy as a function of time with windowing.}
        \label{fig:lversuskenergy}
\end{figure}
\noindent Figure \ref{fig:lversuskenergy} shows that windowing does not work with varying energy, thus a very promising solution strategy from the previous chapter has a limiting factor.\\

\section{Computation time} %%%%%%%%%%%%%%%%%%%%%%%%%%%%%%%%%%%%%%%%%%%%%%%%%%%%%%%%%%%%%
\label{sec:vruntime}
Computation times are somewhat different in this case, compared to the case with constant energy. Windowing is not included since the method did not give the correct answer.
\begin{figure}[H]
        \centering
        \begin{subfigure}[b]{0.3\textwidth}
                \includegraphics[width=\textwidth]{../MATLAB/fig/ltimem.jpg}
                \caption{ Computation time as a function of $m$ without restart. }
                \label{fig:ltimem}
        \end{subfigure}
        ~
        \begin{subfigure}[b]{0.3\textwidth}
                \includegraphics[width=\textwidth]{../MATLAB/fig/ltimek.jpg}
                \caption{ Computation time as a function of $k$ without restart. }
                \label{fig:ltimek}
        \end{subfigure}
        ~
        \begin{subfigure}[b]{0.3\textwidth}
                \includegraphics[width=\textwidth]{../MATLAB/fig/ltimek1.jpg}
                \caption{ Computation time as a function of $n$ without restart. }
                \label{fig:ltimek1}
        \end{subfigure}
        
                \begin{subfigure}[b]{0.3\textwidth}
                \includegraphics[width=\textwidth]{../MATLAB/fig/ltimemr.jpg}
                \caption{ Computation time as a function of $m$ with restart. }
                \label{fig:ltimemr}
        \end{subfigure}
        ~
        \begin{subfigure}[b]{0.3\textwidth}
                \includegraphics[width=\textwidth]{../MATLAB/fig/ltimekr.jpg}
                \caption{ Computation time as a function of $k$ with restart. }
                \label{fig:ltimekr}
        \end{subfigure}
        ~
        \begin{subfigure}[b]{0.3\textwidth}
                \includegraphics[width=\textwidth]{../MATLAB/fig/ltimekr1.jpg}
                \caption{ Computation time as a function of $n$ with restart. }
                \label{fig:ltimekr1}
        \end{subfigure}
        \caption{ A figure of the computation times with and without restart. $n = 200$, $T_s = 100$, $k = 2000$, and $m = 20$. }
        \label{fig:ltime0}
\end{figure}

\noindent Figure \ref{fig:ltime0} shows that when restart is not enabled, both \texttt{KPM} and \texttt{SLM} are slower than \texttt{DM} for $m \geq 100$. \texttt{KPM} is faster than \texttt{DM} after this, while \texttt{SLM} always is slower than \texttt{DM}. 
When restart is enabled, \texttt{KPM} can be faster if $m$ is large, and $k$ is small. \texttt{SLM} is still slower than \texttt{DM}.
Figure \ref{fig:ltimekr1} shows that when restart is not enabled, smaller $n$ gives lower computation times.  