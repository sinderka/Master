\chapter{  }
\section{Code}
All code used to create results in the text can be found on github: \\
\url{https://github.com/sindreka/Master}
\section{Further work}
It could be interesting to see how the error and energy behave with a random Hamiltonian matrix, with a known analytical solution. Specifically \texttt{expm} with restart, and the relation between $n$ and $m$. \\

\noindent Graphics cards are designed to be used on small matrices in parallel \cite{graphics}. When the energy is constant, windowing can be used to ensure convergence. This could then be used to solve non linear Hamiltonian problems in parallel under optimized conditions. \\

\section{Conclusion}
The conclusion is divided in the same two cases that the results were divided into.
%The two different cases discussed in the result sections have behaved quit different. 
%Since the results are divided in two cases it feels natural to divide the conclusion in the same two case.
\subsection{Constant energy} %%%%%%%%%%%%%%%%%%%%%%%%%%%%%%%%%%%%%%%%%%%%%%%%%%%%%%%
The error for both \texttt{KPM} and \texttt{SLM} is found to increase linearly, when a suitable $n$ is used. On large time domains or with small $n$ it was found a sudden increase in error. If restart is enabled the sudden increase is unbounded and exponential, while it is bounded when restart is not enabled.
The sudden increase vanishes when windowing is used. \\

\noindent The energy for both projection methods is increasing very slowly with the length of the time domain, and about as fast as for \texttt{DM}. This suggests that the increase is actually a result of rounding errors, and not a fault in the method. The energy for \texttt{SLM} is always preserved if restart is not enabled.

\noindent Because of \texttt{SLM}'s energy preserving properties it was predicted that it would massively outperform \texttt{KPM}. This has proven to only be partially true, as the error will be equally big for both projection methods. 
%If the energy is the only concern, \texttt{SLM} can achieve this under any case tested in this text. 
If a small error is necessary there are two ways this can be done: Either by restarting or with a larger $n$. If restarting is used \texttt{SLM} looses its energy preserving property, and behaves very similarly to \texttt{KPM}. If $n$ is chosen larger, \texttt{SLM}'s error will decrease, but so will \texttt{KPM}'s error and energy. The two methods will again perform similarly. Thus, in practice \texttt{SLM} has only a small advantage over \texttt{KPM}.\\

\noindent Suitable $n$ is found to be independent of the size of the matrix, and increase linearly with the length of the time domain. If restart is used, $n$ can be about a tenth of the $n$ used without restart.\\

\noindent \texttt{SLM} is near its fastest performance if restart is not enabled. In this case it performed about equal to \texttt{KPM} under the same conditions. \texttt{KPM} can definitely be faster than this if $n$ is chosen optimal, and restarting is used. \\

\noindent If an exact solver is used it is possible to achieve better accuracy with smaller computation time with the projection methods than with trapezoidal rule. The energy for \texttt{SLM} increases linearly in this case, and is constant for \texttt{KPM}.  \\

\noindent I conclude that \texttt{SLM} is better when error is no concern. If the error should be small, \texttt{KPM}'s error and energy is just as small as \texttt{SLM}'s, while being faster. The divergence problem happening on large time domains can be solved with windowing, which also makes computations faster. %Using an exact solver can also increase the accuracy, without any downsides.
\subsection{Varying energy} %%%%%%%%%%%%%%%%%%%%%%%%%%%%%%%%%%%%%%%%%%%%%%%%%%%%%%%%%%
\texttt{SLM} still has an advantage over \texttt{KPM} regarding energy estimation, but only when restart is not enabled. 
Windowing and the exact solvers does not work. Thus, some of the important reasons to use the projection methods are removed. Even with this, both \texttt{SLM} and \texttt{KPM} manages to get error and energy close to \texttt{DM} on small time domains. On large time domains a sudden increase was found. The increase is bounded if restart is not enabled, and unbounded if restart is enabled. This makes the projection methods useless on time domains over a certain size. This restriction might not be too big since the linearly increasing error of \texttt{DM} will make any approximation useless when time domains are above a certain size.\\

\noindent Suitable $n$ is found to be depending linearly on the length of the time domain.\\

\noindent Convergence is more difficult for the projection methods in this case than when the energy was constant. This means that either a larger $n$, or more restarts are necessary to achieve convergence. This results in longer computation times,
making \texttt{SLM} consistently run slower than \texttt{DM}. This eliminates the reason to use \texttt{SLM} in the first place. \texttt{KPM} is barely faster in some cases. Since \texttt{KPM} has a comparable error and energy it may, under some assumptions, be more desirable than \texttt{DM}.
