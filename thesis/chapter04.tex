%%%%%%%%%%%%%%%%%%%%%%%%%%%%%%%%%%%%%%%%%%%%%%%%%%%%%%%%%%%%%%%%%%%%%%%%%%%%%%%%%%%%%%%%%%%%%%%%%%%%%%%%%%%%%%%%%%%%%%
\chapter{Results for test problems with constant energy}%%%%%%%%%%%%%%%%%%%%%%%%%%%%%%%%%%%%%%%%%%%%%%%%%%%%%%%%%%%%%%
\label{sec:constres}
%%%%%%%%%%%%%%%%%%%%%%%%%%%%%%%%%%%%%%%%%%%%%%%%%%%%%%%%%%%%%%%%%%%%%%%%%%%%%%%%%%%%%%%%%%%%%%%%%%%%%%%%%%%%%%%%%%%%%%
This chapter will show how error, energy and computation times changes with the number of restarts $i_r$, the tolerance $\iota$, the time domain $T_s$, the number of steps in time $k$, the number of points in each spacial direction $m$ and the restart variable $n$. We are especially interested in seeing how the energy for \texttt{SLM} behaves, and how \texttt{KPM} and \texttt{SLM} differ. The predictions and proofs made in the theory chapter will also be tested. \\

\noindent Due to the computational complexity of finding an exact solution with \texttt{diag}, which is needed when calculating \texttt{semirandom}'s error, fairly small matrices are used. Unless stated otherwise, assume that $m = 8$, which makes the size of the matrix $\hat{m} = 2(m-2)^2 = 72$, and $k = 20$ per second.

\section{Convergence}%%%%%%%%%%%%%%%%%%%%%%%%%%%%%%%%%%%%%%%%%%%%%%%%%%%%%%%%%%%%%%%%%%%%%%%%%%%%%%%%%%%%%%%
This section shows convergence for the different methods, \texttt{wave} is used since it is necessary to know the analytical solution. The time domain, $T_s = 1$ second for all plots in this section.

\subsection{ Convergence with numerical integration }%%%%%%%%%%%%%%%%%%%%%%%%%%%%%%%%%%%%%%%%%%%%%%%%%%%%%%%%%%%%%%%%%%%%%%%%%%%%%%%%%%%%
\begin{figure}[H]
        \centering
        \begin{subfigure}[b]{0.30\textwidth}
                \includegraphics[width=\textwidth]{../MATLAB/fig/intconv11r.jpg}
                \caption{ Convergence of trapezoidal rule without restart. }
                \label{fig:intconv11r}
        \end{subfigure}
        ~
        \begin{subfigure}[b]{0.30\textwidth}
                \includegraphics[width=\textwidth]{../MATLAB/fig/intconv12r.jpg}
                \caption{ Convergence of forward Euler without restart. }
                \label{fig:intconv12r}
        \end{subfigure}
        \begin{subfigure}[b]{0.30\textwidth}
                \includegraphics[width=\textwidth]{../MATLAB/fig/intconv13r.jpg}
                \caption{ Convergence of midpoint rule without restart. }
                \label{fig:intconv13r}
        \end{subfigure}         
        
        \begin{subfigure}[b]{0.30\textwidth}
                \includegraphics[width=\textwidth]{../MATLAB/fig/intconv11.jpg}
                \caption{ Convergence of trapezoidal rule with restart. }
                \label{fig:intconv11}
        \end{subfigure}
        ~
        \begin{subfigure}[b]{0.30\textwidth}
                \includegraphics[width=\textwidth]{../MATLAB/fig/intconv12.jpg}
                \caption{ Convergence of forward Euler with restart. }
                \label{fig:intconv12}
        \end{subfigure}
        \begin{subfigure}[b]{0.30\textwidth}
                \includegraphics[width=\textwidth]{../MATLAB/fig/intconv13.jpg}
                \caption{ Convergence of midpoint rule with restart. }
                \label{fig:intconv13}
        \end{subfigure}
        
 
\caption{ Convergence for the different integration methods. Notice that forward Euler uses $k^2$ points in time where the other methods use $k$ points in time to obtain the same accuracy. $n=2$ is kept constant for all plots, since smaller $n$ generally gives poorer convergence. For the figures with restart enabled: the tolerance $\iota = 1e-10$. 1 second is simulated. }
\label{fig:intconv}
\end{figure}
\noindent Figure \ref{fig:intconv} shows that all methods converge with the expected rate. 
The trapezoidal rule and the midpoint rule converges quadratically and near identically, the trapezoidal rule and the midpoint coincide on linear constant coefficient problems. So on such Hamiltonian systems they are both symplectic. Since the midpoint rule also is symplectic on problems with non constant coefficients, it will be used when restart is enabled. The trapezoidal rule has a faster run time, and will be used when restart is not enabled due to insignificant differences in error. Both the midpoint and the trapezoidal rule require the solution of one linear system per time step. The linear systems are solved with the function "backslash" in MATLAB. \\

\noindent Forward Euler has a slow computation time due to the larger $k$ needed for the same convergence. It also has a tendency to diverge on longer time intervals and poorly approximates the energy, which is shown Figure \ref{fig:forwardenergy} for trapezoidal rule and forward Euler. The poorly approximated energy is reason enough not to use forward Euler.%This difference in energy is the reason not to use forward Euler any more.\\

\begin{figure}[H]
        \centering
        \begin{subfigure}[b]{0.3\textwidth}
                \includegraphics[width=\textwidth]{../MATLAB/fig/intener11.jpg}
                \caption{ Energy error for trapezoidal rule. }
                \label{fig:fe1}
        \end{subfigure}
        ~
        \begin{subfigure}[b]{0.3\textwidth}
                \includegraphics[width=\textwidth]{../MATLAB/fig/intener12.jpg}
                \caption{ Energy error for forward Euler. }  
				\label{fig:fe2}
        \end{subfigure}                
\caption{ A figure showing the difference in energy between trapezoidal rule and forward Euler. 1 second is simulated and restart is not enabled. }
\label{fig:forwardenergy}
\end{figure}

\subsection{Convergence with an exact solver} %%%%%%%%%%%%%%%%%%%%%%%%%%%%%%%%%%%%%%%%%%%%%%%%%%%%%%%%%%%%%%%%%%%%%%%%%%%%%%%%%%%%%%%%%%%
\label{sec:exactconv}
\begin{figure}[H]
        \centering
        \begin{subfigure}[b]{0.3\textwidth}
                \includegraphics[width=\textwidth]{../MATLAB/fig/exactconvtraperr.jpg}
                \caption{ Convergence of the exact solver without restart. }
                \label{fig:exactconvtraperr}
        \end{subfigure}
%        ~
%        \begin{subfigure}[b]{0.3\textwidth}
%                \includegraphics[width=\textwidth]{../MATLAB/fig/exactconvmiderr.jpg}
%                \caption{ Convergence of the exact solver using \texttt{expm} without restart. }  %
%				\label{fig:exactconvmiderr}
%        \end{subfigure}         
       
\caption{Convergence for exact solvers. \texttt{DM} uses the trapezoidal rule, and is included to show the difference between this figure, and Figure \ref{fig:intconv}. Note that the exactness of the method is for the integration in time, the spacial step-size is kept constant. The expected convergence is therefore still quadratic with $m$. $n = 2$ because taking the matrix exponential is a costly operation and it is harder for the methods to converge when $n$ is small. 1 second is simulated. The exact solver is defined in Table \ref{tab:intcorrect}. }
\label{fig:intexactt}
\end{figure}
Figure \ref{fig:intexactt} shows that the convergence is quadratic for the exact solver. %It might seem strange to include two different exact solvers, but an important difference between the two is shown in Section \ref{sec:resultexpm}.

\section{Convergence with the restart}%%%%%%%%%%%%%%%%%%%%%%%%%%%%%%%%%%%%%%
\label{sec:crestart}
This section will show how error and energy change with the tolerance $\iota$ and the number of restarts $i_r$.
\subsection{Convergence as a function of $\iota$}%%%%%%%%%%%%%%%%%%%%%%%%%%%%%
\begin{figure}[H]
        \centering
        \begin{subfigure}[b]{0.30\textwidth}
                \includegraphics[width=\textwidth]{../MATLAB/fig/lcompareError.jpg}
                \caption{ Error for \texttt{wave}. }
                \label{fig:lcompareError}
        \end{subfigure}
		~
        \begin{subfigure}[b]{0.30\textwidth}
                \includegraphics[width=\textwidth]{../MATLAB/fig/lcompareEnergy.jpg}
                \caption{ Energy error for \texttt{wave}. }
                \label{fig:lcompareEnergy}
        \end{subfigure}
        ~
		\begin{subfigure}[b]{0.30\textwidth}
                \includegraphics[width=\textwidth]{../MATLAB/fig/lcompareIter.jpg}
                \caption{ Number of restarts for \texttt{wave}. }
                \label{fig:lcompareIter}
        \end{subfigure}
        
		\begin{subfigure}[b]{0.30\textwidth}
                \includegraphics[width=\textwidth]{../MATLAB/fig/lcompareErrorw.jpg}
                \caption{ Error for \texttt{semirandom}. \\ .}
                \label{fig:lcompareErrorw}
        \end{subfigure}
		~
		\begin{subfigure}[b]{0.30\textwidth}
                \includegraphics[width=\textwidth]{../MATLAB/fig/lcompareEnergyw.jpg}
                \caption{ Energy error for \texttt{semirandom}. }
                \label{fig:lcompareEnergyw}
        \end{subfigure}
		~
		\begin{subfigure}[b]{0.30\textwidth}
                \includegraphics[width=\textwidth]{../MATLAB/fig/lcompareIterw.jpg}
                \caption{ Number of restarts for \texttt{semirandom}. }
                \label{fig:lcompareIterw}
        \end{subfigure}
        \caption{ These figures show how choosing different tolerances $\iota$ affects the solution. The pictures on the top are for \texttt{wave} and the pictures at the bottom are for \texttt{semirandom}. This plot considers 10 seconds and uses the midpoint rule. Remember that $\text{error}^{\text{diag}}$ is the relative difference between the specified method (eg. \texttt{KPM} or \texttt{SLM}), and an exact solver. \hfill $\neptune$  }
        \label{fig:lcompare}
\end{figure}
Figure \ref{fig:lcompare} shows that if \texttt{wave} is used, there is no reason to use the restart. From Figure \ref{fig:intconv} we conclude that this also is the case for larger $m$. \texttt{wave} will therefore not be used again until Chapter \ref{sec:varyener}. If you are interested in using a Krylov method on the wave equation with constant energy, it can be very well utilized. The largest orthogonal space needed is about $n=2$ (depending on $T_s$, see Section \ref{sec:resultat}), which gives a very low run time.
The rest of this chapter will only consider \texttt{semirandom}, since this better shows the difference between using restarts and not. \\

\noindent Figure \ref{fig:lcompareIterw} shows that the number of restarts are the same for \texttt{KPM} and \texttt{SLM}. Figure \ref{fig:lcompareErrorw} shows that there is little reason to restart after gaining a certain precision, since the changes will not be visible. Based on Figure \ref{fig:lcompareEnergyw} i conclude that $\iota = 1e-6$ is a suitable tolerance. \\

\noindent Figure \ref{fig:lcompareErrorw} shows that both \texttt{KPM} and \texttt{SLM} need to restart to gain a low error. But Figure \ref{fig:lcompareEnergyw} shows that the energy for \texttt{SLM} increases the first times it restarts, before it starts sinking again. This shows that the energy can be accurately estimated by \texttt{SLM} without restart, but not the error. This seems to indicate that the restart procedure ruins the energy preservation property of \texttt{SLM}, but many restarts can better the approximation of the energy. The energy of \texttt{KPM} can also be estimated to machine accuracy, even though its initial energy is quite high.\\

\subsection{Convergence as a function of $i_r$} %%%%%%%%%%%%%%%%%%%%%%%%%%%%%%%%%%%%%%%%%%%%%%

\begin{figure}[H]
        \centering
        \begin{subfigure}[b]{0.3\textwidth}
                \includegraphics[width=\textwidth]{../MATLAB/fig/cierr2.jpg}
                \caption{ Error as a function of $i_r$. \\.}
                \label{fig:cierr2}
        \end{subfigure}
        ~
		\begin{subfigure}[b]{0.3\textwidth}
                \includegraphics[width=\textwidth]{../MATLAB/fig/ciene2.jpg}
                \caption{ Energy error as a function of $i_r$. }
                \label{fig:ciene2}
        \end{subfigure}    

        \begin{subfigure}[b]{0.3\textwidth}
                \includegraphics[width=\textwidth]{../MATLAB/fig/cierr1.jpg}
                \caption{ Error as a function of time with constant $i_r = 10$. }
                \label{fig:cierr1}
        \end{subfigure}
        ~
		\begin{subfigure}[b]{0.3\textwidth}
                \includegraphics[width=\textwidth]{../MATLAB/fig/ciene1.jpg}
                \caption{ Energy error a function of time with constant $i_r = 10$. }
                \label{fig:ciene1}
        \end{subfigure}
        \caption{ The top figures show how the error and energy change as a function of the number of restarts $i_r$. The bottom pictures show how error and energy behave with increasing time domain, with constant $i_r = 10$, and $T_s = 10$.}
        \label{fig:ci}
\end{figure}
\noindent Figure \ref{fig:cierr2} and \ref{fig:ciene2} show that for \texttt{KPM}(6) and \texttt{SLM}(6) it is necessary to perform more than 4 restarts to gain accuracy w.r.t. error. After 8 restarts the change in the solution is too small to observe. This shows why it is wise to use $\iota$ and not $i_r$ as convergence criterion. If $\iota$ is used, the Krylov methods can perform the exact number of iterations needed to converge, with any $n$ and $T_s$. If $i_r$ is used as convergence criterion, it needs to be changed with $n$ and $T_s$.
Figure \ref{fig:cierr1} shows that the error increases linearly with time for the first few points, as predicted in \ref{sec:DM}.
Figure \ref{fig:ciene1} shows that the energy increases, \texttt{SLM} shows no energy preserving property in this case.

\section{How to choose the restart variable $n$}%%%%%%%%%%%%%%%%%%%%%%%%%%%%%%%%%%%%%%%%%%%%%%%%%%%%%%%%%%%%%%
\label{sec:resultat}
This section will look at how $n$ can be chosen to avoid unnecessary restart, and still give satisfactory results. Figure \ref{fig:lcompareErrorw} and \ref{fig:lcompareEnergyw} shows that the energy and error behaves quite similarly, except when $i_r = 0$ for \texttt{SLM}. If only energy preservation is important, there is no need to use \texttt{KPM}, or to wisely choose an $n$. This section will therefore only consider the error.  How $n$ should change with the size of the matrix $A$ (which is given by $\hat{m} = 2(m-2)^2$, see Section \ref{sec:wave} for a more thorough explanation), and the length of the time domain $T_s$, is explored. We assume that all cases tested here are independent. 

\subsection{Restart variable as a function of $m$} %%%%%%%%%%%%%%%%%%%%%%%%%%%%%%%%%%%%%%%%%%%%%%%%
\begin{figure}[H]
        \centering
        \begin{subfigure}[b]{0.3\textwidth}
                \includegraphics[width=\textwidth]{../MATLAB/fig/nerrorwA.jpg}
                \caption{ Error for \texttt{KPM} without restart. }
                \label{fig:nerrorwA}
        \end{subfigure}
        ~
        \begin{subfigure}[b]{0.3\textwidth}
                \includegraphics[width=\textwidth]{../MATLAB/fig/nerrorwS.jpg}
                \caption{ Error for \texttt{SLM} without restart. }
                \label{fig:nerrorwS}
        \end{subfigure}
        
		\begin{subfigure}[b]{0.3\textwidth}
                \includegraphics[width=\textwidth]{../MATLAB/fig/reserrA.jpg}
                \caption{ Error for \texttt{KPM} with restart. }
                \label{fig:reserrA}
        \end{subfigure}
		~
		\begin{subfigure}[b]{0.3\textwidth}
                \includegraphics[width=\textwidth]{../MATLAB/fig/reserrS.jpg}
                \caption{ Error for \texttt{SLM} with restart. }
                \label{fig:reseneS}
        \end{subfigure}
        \caption{ The pictures show which $n$ gives convergence for different $m$, where the size of the matrix $A$, is given by $\hat{m} = 2(m-2)^2$. The top pictures are without restart, and the bottom pictures are with restart. 10 seconds is simulated. }
        \label{fig:n}
\end{figure}
\noindent Figure \ref{fig:nerrorwA} and \ref{fig:nerrorwS} show that $n$ can be choose independently of $m$, as long as $n \geq 40$ when restart is not enabled. Figure \ref{fig:reserrA} and \ref{fig:reseneS} show that if restart is enabled, a good approximation of the solution can be found with any $n$, for any $m$. The reason for the independence between $m$ and $n$ can be due to the structure of the matrix.%, and is not a rule for general Hamiltonian matrices. \\

%\noindent The reason for the large error is that this is the approximation possible with the number of steps in time $k=20$ that is used.

\subsection{Restart variable as a function of $T_s$} %%%%%%%%%%%%%%%%%%%%%%%%%%%%%%%%%%%%%%%%%%%%%%%%%%

\begin{figure}[H]
        \centering
        \begin{subfigure}[b]{0.3\textwidth}
                \includegraphics[width=\textwidth]{../MATLAB/fig/terrorwA.jpg}
                \caption{ Error for \texttt{KPM} without restart. }
                \label{fig:terrorwA}
        \end{subfigure}
		~
		\begin{subfigure}[b]{0.3\textwidth}
                \includegraphics[width=\textwidth]{../MATLAB/fig/terrorwS.jpg}
                \caption{ Error for \texttt{SLM} without restart. }
                \label{fig:terrorwS}
        \end{subfigure}
        
        \begin{subfigure}[b]{0.3\textwidth}
                \includegraphics[width=\textwidth]{../MATLAB/fig/terrorwAr.jpg}
                \caption{ Error for \texttt{KPM} with restart. }
                \label{fig:terrorwAr}
        \end{subfigure}
		~
		\begin{subfigure}[b]{0.3\textwidth}
                \includegraphics[width=\textwidth]{../MATLAB/fig/terrorwSr.jpg}
                \caption{ Error for \texttt{SLM} with restart. }
                \label{fig:terrorwSr}
        \end{subfigure}        
        
        \caption{ The pictures show which $n$ gives convergence for different $T_s$. The top plots are without restart, and the bottom pictures are with restart.}
        \label{fig:rest}
\end{figure}
Figure \ref{fig:terrorwA} and \ref{fig:terrorwS} show that the length of the time domain is affecting the optimal choice of $n$ if restart is not enabled. In this case $n$ should increase linearly with $T_s$. $n$ and $T_s$ are independent when restart is enabled. The lines are not overlapping due to the linear increase in error over time.\\

\noindent \texttt{SLM} and \texttt{KPM} behave very similarly in all pictures shown in this section, thus the values of $n$ will be the same for the different methods. Based on Figure \ref{fig:n} and \ref{fig:rest}, this text will consider $n = 40$ when restart is not enabled, and $n = 8$ when restart is enabled. 

\section{ Energy and error }%%%%%%%%%%%%%%%%%%%%%%%%%%%%%%%%%%%%%%%%%%%%%%%%%%%%%
\label{sec:resultconsterergy}
This section will show how the error and energy of \texttt{SLM}, \texttt{KPM} and \texttt{DM} changes as a function of time. Restarts and windowing will also be tested. 

\subsection{ Energy and error as a function of time } %%%%%%%%%%%%%%%%%%%%%%%%%%%%%%%%%%%%%%%%%%%%%%%%%%%%%%%%%%%%%%%%%%%%%%%%%%%%%%%
In this case, the midpoint rule is used if the restart is enabled, while the trapezoidal rule is used if the restart is not enabled. Windowing is not considered.

\begin{figure}[H]
        \centering
        \begin{subfigure}[b]{0.3\textwidth}
                \includegraphics[width=\textwidth]{../MATLAB/fig/longtime2err.jpg}
                \caption{ Error without restart. \\. }
                \label{fig:longtime2err}
        \end{subfigure}
        ~
        \begin{subfigure}[b]{0.3\textwidth}
                \includegraphics[width=\textwidth]{../MATLAB/fig/longtime2ene.jpg}
                \caption{ Energy error without restart. }
                \label{fig:longtime8err}
        \end{subfigure}
        
        \begin{subfigure}[b]{0.3\textwidth}
                \includegraphics[width=\textwidth]{../MATLAB/fig/longtime2rerr.jpg}
                \caption{ Error with restart. }
                \label{fig:longtime2rerr}
        \end{subfigure}
        ~
        \begin{subfigure}[b]{0.3\textwidth}
                \includegraphics[width=\textwidth]{../MATLAB/fig/longtime2rene.jpg}
                \caption{ Energy error with restart. }
                \label{fig:longtime8rerr}
        \end{subfigure}
        ~
        \begin{subfigure}[b]{0.3\textwidth}
                \includegraphics[width=\textwidth]{../MATLAB/fig/longtime2rite.jpg}
                \caption{ Number of restarts. }
                \label{fig:longtime2rene}
        \end{subfigure}
        \caption{ The figures show how the error and energy changes with as a function of time. The top pictures are without restart, and the bottom pictures are with restart and $\iota = 1e-6$. }
        \label{fig:SLMenergyerror0}
\end{figure}
\noindent Figure \ref{fig:SLMenergyerror0} shows that the error for all methods are very similar, and increases linearly. Figure \ref{fig:longtime8err} shows that \texttt{KPM}s energy increases very fast, while \texttt{SLM} and \texttt{DM}s energy remains constant. When restart is enabled, \texttt{SLM} and \texttt{KPM}s energy are very similar, and a little worse than \texttt{SLM}s energy without restart. The number of restarts increases as a function of $T_s$. 
\subsection{Energy and error as a function of time for windowing.} %%%%%%%%%%%%%%%%%%%%%%%%%%%%%%%%%%%%%%%%%%%%
Windowing (see Section \ref{sec:windu}) is now used together with the Krylov methods. The $K$ windows each have a length $1$ second. The window is chosen fairly small based on the energies in Figure \ref{fig:longtime8err} and \ref{fig:longtime8rerr}, in these figures, the energy increases when $T_s$ becomes too large.

\begin{figure}[H]
        \centering
        \begin{subfigure}[b]{0.3\textwidth}
                \includegraphics[width=\textwidth]{../MATLAB/fig/Kversuskerror0.jpg}
                \caption{ Error without restart. \\.}
                \label{fig:Kversuskerror0}
        \end{subfigure}
		~
		\begin{subfigure}[b]{0.3\textwidth}
                \includegraphics[width=\textwidth]{../MATLAB/fig/Kversuskenergy0.jpg}
                \caption{ Energy error without restart. }
                \label{fig:Kversuskenergy0}
        \end{subfigure}

        \begin{subfigure}[b]{0.3\textwidth}
                \includegraphics[width=\textwidth]{../MATLAB/fig/Kversuskerror.jpg}
                \caption{ Error with restart. }
                \label{fig:Kversuskerror}
        \end{subfigure}
		~
		\begin{subfigure}[b]{0.3\textwidth}
                \includegraphics[width=\textwidth]{../MATLAB/fig/Kversuskenergy.jpg}
                \caption{ Energy error with restart. }
                \label{fig:Kversuskenergy}
        \end{subfigure}        
        
        \caption{ The figures show how the error and energy change over time when windowing is used. The top pictures are without restart, and the bottom pictures are with restart and $\iota = 1e-6$. }
        \label{fig:Kversusk}
\end{figure}
\noindent Figure \ref{fig:Kversusk} shows that when windowing is enabled, the increasing energy of \texttt{KPM} that occurred on Figure \ref{fig:longtime8err} disappears. If restarting is not  enabled, the error increases linearly, and the energy is preserved for all methods. The energy is slowly increasing when restart is enabled. The error and energy with windowing is comparable to Figure \ref{fig:SLMenergyerror0}. This makes windowing an interesting idea.
\subsection{Behavior of energy and error on long time domains} %%%%%%%%%%%%%%%%%%%%%%%%%%%%%%%%%%%%%%%%%%%%%%%%%%%%%
In this section it is shown what happens when the time domain becomes to large for the Krylov methods to converge.
\label{sec:longtime}
\begin{figure}[H]
        \centering
        \begin{subfigure}[b]{0.3\textwidth}
                \includegraphics[width=\textwidth]{../MATLAB/fig/vlongerr.jpg}
                \caption{ Error without restart. \\. }
                \label{fig:vlongerr}
        \end{subfigure}
		~
		\begin{subfigure}[b]{0.3\textwidth}
                \includegraphics[width=\textwidth]{../MATLAB/fig/vlongene.jpg}
                \caption{ Energy error without restart. }
                \label{fig:vlongene}
        \end{subfigure}
        
        \begin{subfigure}[b]{0.3\textwidth}
                \includegraphics[width=\textwidth]{../MATLAB/fig/vlongerrr.jpg}
                \caption{ Error with restart. }
                \label{fig:vlongerrr}
        \end{subfigure}
		~
		\begin{subfigure}[b]{0.3\textwidth}
                \includegraphics[width=\textwidth]{../MATLAB/fig/vlongener.jpg}
                \caption{ Energy error with restart. }
                \label{fig:vlongener}
        \end{subfigure}
        \caption{ This figure shows how different methods perform on a long time domain. }
        \label{fig:vlong}
\end{figure}
\noindent Figure \ref{fig:vlongerr} shows that, without restart, the increase in error is linear until the numerical value for the errors is about the same size as the numerical value of the exact solution, ei. $\sim 1$. The energy for \texttt{KPM} increase to about $1$ before flattening out, while the energy for \texttt{SLM} and \texttt{DM} is near constant. Actually the energy for \texttt{SLM} and \texttt{DM} is near constant for all values of $T_s$, $m$, and $n$.
Figure \ref{fig:vlongerrr} and \ref{fig:vlongener} shows that when restart is enabled, the energy and error for the Krylov methods diverges. This means: Do not use a Krylov method on a long time domain when restart is enabled, without increasing $n$. \\

\noindent Figure \ref{fig:vlongerrrK} and \ref{fig:vlongerrrKr} show that if windowing is used, the trend of increasing error continues as it did on Figure \ref{fig:Kversusk}, until it becomes constant, around 100 seconds. Figure \ref{fig:vlongerrrK} shows that \texttt{KPM}s energy is poorly approximated if restart is not enabled. When restart is enabled the energy increases slowly for all methods, suggesting the numerical round off errors are to blame. Windowing can effectively be used to avoid the divergence happening on long time domains when restart is enabled.

\begin{figure}[H]
        \centering
		\begin{subfigure}[b]{0.3\textwidth}
                \includegraphics[width=\textwidth]{../MATLAB/fig/vlongerrrK.jpg}
                \caption{ Error for windowing without restart. }
                \label{fig:vlongerrrK}
        \end{subfigure}
        ~
		\begin{subfigure}[b]{0.3\textwidth}
                \includegraphics[width=\textwidth]{../MATLAB/fig/vlongenerK.jpg}
                \caption{ Energy error for windowing without restart. }
                \label{fig:vlongenerK}
        \end{subfigure}             
        
        	\begin{subfigure}[b]{0.3\textwidth}
                \includegraphics[width=\textwidth]{../MATLAB/fig/vlongerrrKr.jpg}
                \caption{ Error for windowing with restart. }
                \label{fig:vlongerrrKr}
        \end{subfigure}
        ~
		\begin{subfigure}[b]{0.3\textwidth}
                \includegraphics[width=\textwidth]{../MATLAB/fig/vlongenerKr.jpg}
                \caption{ Energy error for windowing with restart. }
                \label{fig:vlongenerKr}
        \end{subfigure}      
        \caption{  This figure shows how windowing performs on a very long time domain. }
        \label{fig:windowingc}
\end{figure}


%\subsection{Residual energy}%%%%%%%%%%%%%%%%%%%%%%%%%%%%%%%%%%%%%%%%%%%%%%%%%%%%%%%%%%%%%
%\label{sec:residualenergy}
%\begin{figure}[H]
%        \centering
%        \begin{subfigure}[b]{0.45\textwidth}
%                \includegraphics[width=\textwidth]{../MATLAB/fig/SLMpes.jpg}
%        \end{subfigure}
%		
%        \caption{ A plot of $\mathcal{H}_3$ and $\mathcal{H}_4$ for different time domains. }
%        \label{fig:SLMpes}
%\end{figure}
%$\mathcal{H}_3$ and $\mathcal{H}_4$ are very similar, as is predicted in the theory section. The difference between the two is machine accuracy until $T_s$ gets to big (around 10 seconds). \\
\section{Energy and error with exact solvers} %%%%%%%%%%%%%%%%%%%%%%%%%%%%%%%%%%%%%%%%%%%%%%%%%%%%%%%%%%%%%%%%%%%%%%%%%%%%%%%%%%%%%%%%%%%
\label{sec:diag}
This section will show how using an exact solver can improve the error and energy, compared to the trapezoidal rule. It is also shown how restarting with midpoint rule changes the energy and error after using an exact solver, and how windowing behaves with an exact solver. %Remember that it is not possible to use an exact solver with the restart due to the source term in the error equation. \\

\subsection{Exact solver vs. numerical integration}%%%%%%%%%%%%%%%%%%%%%%%%%%%%%%%%
\begin{figure}[H]
        \centering
        \begin{subfigure}[b]{0.45\textwidth}
                \includegraphics[width=\textwidth]{../MATLAB/fig/ideaerr20.jpg}
                \caption{ Error as a function of time. }
                \label{fig:ideaerr20}
        \end{subfigure}%
        ~
        \begin{subfigure}[b]{0.45\textwidth}
                \includegraphics[width=\textwidth]{../MATLAB/fig/ideaener20.jpg}
                \caption{ Energy error as a function of time. }
                \label{fig:ideaener20}
        \end{subfigure}
       
        \caption{ A figure showing how the error and energy changes over time when an exact solver (\texttt{diag}) is used. Restart is not enabled, and \texttt{DM} uses the trapezoidal rule.  }
        \label{fig:idea0}
\end{figure}

\noindent Figure \ref{fig:ideaerr20} shows how the Krylov methods compare to each other with an exact solver and with the trapezoidal rule. \texttt{diag} makes it possible for the Krylov methods to be within machine accuracy of the exact solution, thou only for a short time. After about 11 second the difference in error between \texttt{trap} and \texttt{diag} is minimal, in this case the leading error is caused by the Krylov method, and not the exact solver. 
Figure \ref{fig:ideaener20} shows that using an exact solver does not change the energy, compared with the trapezoidal rule.





\subsection{Exact solver with restart} %%%%%%%%%%%%%%%%%%%%%%%%%%%%%%%%%%%%%%%%%%
\begin{figure}[H]
		\begin{subfigure}[b]{0.45\textwidth}
                \includegraphics[width=\textwidth]{../MATLAB/fig/winduerr.jpg}
                \caption{ Error with and without restart. }
                \label{fig:winduerr}
        \end{subfigure}%
        ~
        \begin{subfigure}[b]{0.45\textwidth}
                \includegraphics[width=\textwidth]{../MATLAB/fig/winduene.jpg}
                \caption{ Energy error with and without restart. }
                \label{fig:winduene}
        \end{subfigure}
\caption{A figure showing how restarting after using an exact solver changes the error and energy. $i_r = 1$ means that restart is not used, while $i_r > 1$ means that restart is used. }        
\label{fig:windu}
\end{figure}

\noindent Figure \ref{fig:windu} shows that the difference between restarting and not is quite small when an exact solver is used. It seems that restarting gives better results at the end of the time domain, while not restarting gives better results at the beginning of the time domain. However, this effect is not large enough to conclude anything definite.

\subsection{Exact solver with windowing}%%%%%%%%%%%%%%%%%%%%%%%%%%%%%%%%%%%%%%%%%%%%%
\begin{figure}[H]
		\begin{subfigure}[b]{0.45\textwidth}
                \includegraphics[width=\textwidth]{../MATLAB/fig/restarterr.jpg}
                \caption{ Error with and without restart. }
                \label{fig:restarterr}
        \end{subfigure}
        ~
        \begin{subfigure}[b]{0.45\textwidth}
                \includegraphics[width=\textwidth]{../MATLAB/fig/restartene.jpg}
                \caption{ Energy error with and without restart. }
                \label{fig:restartene}
        \end{subfigure}
\caption{A figure showing how windowing with an exact solver changes the error and the energy. $i_r = 1$ means that restart is not used, while $i_r > 1$ means that restart is used. }        
\label{fig:restart}
\end{figure}

\noindent Figure \ref{fig:restarterr} shows that the error is linearly increasing, this makes windowing with an exact solver better than just the exact solver after 10 seconds, but before this, the exact solver much better. Figure \ref{fig:restartene} shows that the energy becomes poorly approximated after a few seconds, only \texttt{SLM}s energy with restart is constant and near machine accuracy. %Restarting gives a larger error and a smaller energy, for all methods.

%\subsection{MATLABs \texttt{expm} function} \label{sec:resultexpm} %%%%%%%%%%%%%%%%%%%%%%%%%%%%%%%%%%%%%%%%%%%%%%%%%%%%%%%%%%%%%%%%%%%%%%%%%%%%%
%This section will show the reason for using \texttt{diag} instead of \texttt{expm} (see  Table \ref{tab:intcorrect}).
%\begin{figure}[H]
%        \centering
%		\begin{subfigure}[b]{0.45\textwidth}
%                \includegraphics[width=\textwidth]{../MATLAB/fig/expmAerr.jpg}
%                \caption{ Error. }
%                \label{fig:expmSerr}
%        \end{subfigure}%
%        ~
%        \begin{subfigure}[b]{0.45\textwidth}
%                \includegraphics[width=\textwidth]{../MATLAB/fig/expmAener.jpg}
%                \caption{ Energy error. }
%                \label{fig:expmSener}
%        \end{subfigure}
%        \caption{A figure showing the difference in error and energy for the different exact integration methods. Restart is not enabled. }
%        \label{fig:expm}
%\end{figure}
%\noindent The error for the two exact solvers are identical, but there is a big difference in energy. The methods have a constant energy with \texttt{diag}, while the energy increases fast when \texttt{expm} is used.\\

\section{Energy in the transformations}%%%%%%%%%%%%%%%%%%%%%%%%%%%%%%%%%%%%%%%%%%%%
\label{sec:transf}
This section will show how the energy changes when transforming from $z_n(t)$ to $u_n(t)$. For \texttt{KPM} $u_n(t) = V_n z_n(t)$, and for \texttt{SLM} $u_n = S_n z_n(t)$. The difference is the matrix $V_n$ and $S_n$, where $V_n$ is an orthonormal matrix, and $S_n$ is a symplectic matrix. It is predicted that the transformation with $S_n$ would be energy preserving, no predictions is made regarding the energy for the transformation with $V_n$. For more information see Section \ref{sec:KPM} and \ref{sec:SLM}. 

\begin{figure}[H]
        \centering
        \begin{subfigure}[b]{0.3\textwidth}
                \includegraphics[width=\textwidth]{../MATLAB/fig/energswA.jpg}
                \caption{ Energy error for \texttt{KPM}(40). }
                \label{fig:energyswA}
        \end{subfigure}
        ~
		\begin{subfigure}[b]{0.3\textwidth}
                \includegraphics[width=\textwidth]{../MATLAB/fig/energswS.jpg}
                \caption{ Energy error for \texttt{SLM}(40). }
                \label{fig:energswS}
        \end{subfigure}        
        \caption{ The figures show how the transformation between $z(t)$ and $u(t)$ changes the energy. Restart is not enabled.}
        \label{fig:energs}
\end{figure}
\noindent For \texttt{KPM} there is a huge discrepancy between the energy $\mathcal{H}_2$ for $z_{n}(t)$, and $\mathcal{H}_1$ $u_{n}(t)$, see Section \ref{sec:energy} for more information about the energy. For \texttt{SLM} there is no difference between the two. This shows the transformation with $S_n$ is symplectic. It is worth mentioning that $S_n J_{\hat{m}}S_n - J_n \approx 1e-16$ and $V_n^\top V_n - I_n \approx 1e-11$, when $m = 20$ and $n = 200$.

\section{Computation time}%%%%%%%%%%%%%%%%%%%%%%%%%%%%%%%%%%%%%%%%%%%%%%%%%%%%%%%%%%%%%%%%%%%%%%%%%%%%%%%%%%%%%%%%%
\label{sec:cruntime}
This section will compare computation time for the different methods discussed. Matrices with the size $m = 8$ are far too small to show optimal computation times for the Krylov methods. Therefore assume that $m = 20$. $k = 20$ is still suitable. The restart variables for this size is found using Section \ref{sec:resultat} with $T_s = 100$, this gives $n = 200$ when restart is not enabled, and $n = 20$ when restart is enabled. An additional reason to pick $n = 20$ when restart is not enabled is that the number of iterations decreases as $n$ increases, as you will see,  $n \sim m$ is a good rule for optimal computation times.

\subsection{Computation time for the numerical integration} \label{sec:naive}
In this case, the trapezoidal and the midpoint rule are used, without windowing. Computation time for different $m$, $k$ and $n$ is shown.
\begin{figure}[H]
        \centering
        \begin{subfigure}[b]{0.3\textwidth}
                \includegraphics[width=\textwidth]{../MATLAB/fig/timem.jpg}
                \caption{ Computation time as a function of $m$ without restart. }
                \label{fig:timem}
        \end{subfigure}
        ~
        \begin{subfigure}[b]{0.3\textwidth}
                \includegraphics[width=\textwidth]{../MATLAB/fig/timek.jpg}
                \caption{ Computation time as a function of $k$ without restart. }
                \label{fig:timek}
        \end{subfigure}
        ~
        \begin{subfigure}[b]{0.3\textwidth}
                \includegraphics[width=\textwidth]{../MATLAB/fig/timek1.jpg}
                \caption{ Computation time as a function of $n$ without restart. }
                \label{fig:timek1}
        \end{subfigure}
        
        \begin{subfigure}[b]{0.3\textwidth}
                \includegraphics[width=\textwidth]{../MATLAB/fig/timemr.jpg}
                \caption{ Computation time as a function of $m$ with restart. }
                \label{fig:timemr}
        \end{subfigure}
        ~
        \begin{subfigure}[b]{0.3\textwidth}
                \includegraphics[width=\textwidth]{../MATLAB/fig/timekr.jpg}
                \caption{ Computation time as a function of $k$ with restart. }
                \label{fig:timekr}
        \end{subfigure}
        ~
        \begin{subfigure}[b]{0.3\textwidth}
                \includegraphics[width=\textwidth]{../MATLAB/fig/timekr1.jpg}
                \caption{ Computation time as a function of $n$ with restart. }
                \label{fig:timekr1}
        \end{subfigure}          
        
        \caption{ A figure of computation times with and without restart for different $m$, $k$ and $n$. Assume that $n = 200$, $T_s = 100$, $k = 2000$, and $m = 20$. }
        \label{fig:time0}
\end{figure}
\noindent Figure \ref{fig:time0} shows that the computation time for all methods, except \texttt{KPM} with restart, increases linearly with $k$. \texttt{KPM}'s computation time seems to be increasing quadratically with $k$. The computation time for \texttt{DM} increases quadratically with $m$. \texttt{SLM} is faster than \texttt{DM} if restart is not used, but slower if restart is used. The difference between \texttt{KPM} with and without restart is minimal, and is similar to \texttt{SLM}'s computation time without restart.  Both \texttt{KPM} and \texttt{SLM} are fastest without restart. \texttt{KPM} can also be fast with restart if $m$ is large, $k$ is small, and $n$ is well chosen. Figure \ref{fig:timekr1} and \ref{fig:timek1} show that choosing $n$ optimal is very important for the Krylov methods to achieve good run times. $n \approx m$ is a good rule when restart is enabled, thou this rule is less important for \texttt{SLM}. When restart is not enabled, smaller $n$ gives lower computation times. 
\subsection{ Computation time with windowing}
In this case, the trapezoidal and the midpoint rule are used, with windowing. Computation time for different $m$ and $k$ is shown.
\begin{figure}[H]
        \centering
        \begin{subfigure}[b]{0.3\textwidth}
                \includegraphics[width=\textwidth]{../MATLAB/fig/timemt.jpg}
                \caption{ Computation time as a function of $m$ without restart. }
                \label{fig:timemt}
        \end{subfigure}
        ~
        \begin{subfigure}[b]{0.3\textwidth}
                \includegraphics[width=\textwidth]{../MATLAB/fig/timekt.jpg}
                \caption{ Computation time as a function of $k$ without restart. }
                \label{fig:timekt}
        \end{subfigure}

        \begin{subfigure}[b]{0.3\textwidth}
                \includegraphics[width=\textwidth]{../MATLAB/fig/timemtr.jpg}
                \caption{ Computation time as a function of $m$ with restart. }
                \label{fig:timemtr}
        \end{subfigure}
        ~
        \begin{subfigure}[b]{0.3\textwidth}
                \includegraphics[width=\textwidth]{../MATLAB/fig/timektr.jpg}
                \caption{ Computation time as a function of $k$ with restart. }
                \label{fig:timektr}
        \end{subfigure}        
        
        \caption{ A figure of the computation times when windowing is used. Assume that $n = 20$, $T_s = 100$, $k = 20$ per second, and $m = 20$. }
        \label{fig:time2}
\end{figure}
\noindent Figure \ref{fig:time2} shows that the computation time for all methods increase linearly with time. There is a big difference between restarting and not for \texttt{SLM}, where \texttt{SLM} should avoid the restart, since it then is slower than \texttt{DM}. For \texttt{KPM} the difference between restarting and not is minimal, and it is always faster than \texttt{DM}. \\

\noindent One great thing with the computation times with windowing for \texttt{KPM} is that, even with restart, the computation times increase linearly with time, making \texttt{KPM} faster than \texttt{DM} for all values of $k$ and $m$.
\subsection{Computation time with \texttt{diag}}
\begin{figure}[H]
        \centering
        \begin{subfigure}[b]{0.3\textwidth}
                \includegraphics[width=\textwidth]{../MATLAB/fig/timeme.jpg}
                \caption{ Computation time as a function of $m$ without restart. }
                \label{fig:timeme}
        \end{subfigure}
        ~
        \begin{subfigure}[b]{0.3\textwidth}
                \includegraphics[width=\textwidth]{../MATLAB/fig/timeke.jpg}
                \caption{ Computation time as a function of $k$ without restart. }
                \label{fig:timeke}
        \end{subfigure}
        
        \begin{subfigure}[b]{0.3\textwidth}
                \includegraphics[width=\textwidth]{../MATLAB/fig/timemer.jpg}
                \caption{ Computation time as a function of $m$ with restart. }
                \label{fig:timemer}
        \end{subfigure}
        ~
        \begin{subfigure}[b]{0.3\textwidth}
                \includegraphics[width=\textwidth]{../MATLAB/fig/timeker.jpg}
                \caption{ Computation time as a function of $k$ with restart. }
                \label{fig:timeker}
        \end{subfigure}
        \caption{ A figure of the computation times with \texttt{diag}. Assume that $n = 20$, $T_s = 100$, $k = 2000$, and $m = 20$. }
        \label{fig:time3}
\end{figure}
Figure \ref{fig:time3} shows that using an exact solver is equally as fast as using any of the other approaches discussed. This, together with the small error, makes exact solvers very desirable on small time domains. \texttt{SLM} with restart has the same problem as in the other cases, where the restart makes it very slow. \texttt{KPM} also has the same problem as before, where computation time increases faster than linear when restart is enabled. \\

\noindent No plots of time consumption with windowing and an exact solver is shown, because the results are very similar to the ones shown in Figure \ref{fig:time2}. 