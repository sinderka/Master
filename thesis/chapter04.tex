\section{Divide the time domain}%%%%%%%%%%%%%%%%%%%%%%%%%%%%%%%%%%%%%%%%%%%%%%%%%%%%%%%%%%%%%%%%%%%%%%%%%%%%%%%%%%%%%%

One of the major problems with the orthogonalisation methods is that when dealing with large time intervals it might not converge, and the restarts does not help at all. So lets try something different. In this section the time domain is divided in smaller pieces so that the restart can be used without the energy and error diverging, as described in section \ref{sec:timedomain}. Since using the orthogonalisation algorithms might be time consuming that will also be something to look at. 

\subsection{Constant energy}%%%%%%%%%%%%%%%%%%%%%%%%%%%%%%%%%%%%%%%%%%%%%%%%%%%%%%%%%%%%%%%%%%%%%%%%%%%%%%%%%%%%%%%%%%
\begin{figure}[H]
        \centering
		\begin{subfigure}[b]{0.3\textwidth}
                \includegraphics[width=\textwidth]{../MATLAB/fig/Kversusktime0.jpg}
                \caption{  }
                \label{fig:Kversusktime0}
        \end{subfigure}
        ~
        \begin{subfigure}[b]{0.3\textwidth}
                \includegraphics[width=\textwidth]{../MATLAB/fig/Kversuskerror0.jpg}
                \caption{  }
                \label{fig:Kversuskerror0}
        \end{subfigure}
        ~
        \begin{subfigure}[b]{0.3\textwidth}
                \includegraphics[width=\textwidth]{../MATLAB/fig/Kversuskenergy0.jpg}
                \caption{  }
                \label{fig:Kversuskenergy0}
        \end{subfigure}
        
        \begin{subfigure}[b]{0.3\textwidth}
                \includegraphics[width=\textwidth]{../MATLAB/fig/Kversusktime.jpg}
                \caption{  }
                \label{fig:Kversusktime}
        \end{subfigure}
        ~
        \begin{subfigure}[b]{0.3\textwidth}
                \includegraphics[width=\textwidth]{../MATLAB/fig/Kversuskerror.jpg}
                \caption{  }
                \label{fig:Kversuskerror}
        \end{subfigure}
        ~
        \begin{subfigure}[b]{0.3\textwidth}
                \includegraphics[width=\textwidth]{../MATLAB/fig/Kversuskenergy.jpg}
                \caption{  }
                \label{fig:Kversuskenergy}
        \end{subfigure}
        \caption{Figure showing how the different methods perform with different distribution of $K$ and $k$. Notice that the product $Kk$ is constant. }
        \label{fig:Kversusk}
\end{figure}
DM is not affected much by this change, with energy, error and time comsuption nearly constant. But for SLPM and KPM 

\subsection{Varying energy}%%%%%%%%%%%%%%%%%%%%%%%%%%%%%%%%%%%%%%%%%%%%%%%%%%%%%%%%%%%%%%%%%%%%%%%%%%%%%%%%%%%%%%%%%%%
\begin{figure}[H]
        \centering
		\begin{subfigure}[b]{0.3\textwidth}
                \includegraphics[width=\textwidth]{../MATLAB/fig/Kversusktime20.jpg}
                \caption{  }
                \label{fig:Kversusktime20}
        \end{subfigure}
        ~
        \begin{subfigure}[b]{0.3\textwidth}
                \includegraphics[width=\textwidth]{../MATLAB/fig/Kversuskerror20.jpg}
                \caption{  }
                \label{fig:Kversuskerror20}
        \end{subfigure}
        ~
        \begin{subfigure}[b]{0.3\textwidth}
                \includegraphics[width=\textwidth]{../MATLAB/fig/Kversuskenergy20.jpg}
                \caption{  }
                \label{fig:Kversuskenergy20}
        \end{subfigure}        
        
        \begin{subfigure}[b]{0.3\textwidth}
                \includegraphics[width=\textwidth]{../MATLAB/fig/Kversusktime2.jpg}
                \caption{  }
                \label{fig:Kversusktime2}
        \end{subfigure}
        ~
        \begin{subfigure}[b]{0.3\textwidth}
                \includegraphics[width=\textwidth]{../MATLAB/fig/Kversuskerror2.jpg}
                \caption{  }
                \label{fig:Kversuskerror2}
        \end{subfigure}
        ~
        \begin{subfigure}[b]{0.3\textwidth}
                \includegraphics[width=\textwidth]{../MATLAB/fig/Kversuskenergy2.jpg}
                \caption{  }
                \label{fig:Kversuskenergy2}
        \end{subfigure}
        \caption{Figure showing how the different methods perform with different distribution of $K$ and $k$. Notice that the product $Kk$ is constant. }
        \label{fig:Kversusk2}
\end{figure}
Figure \ref{fig:Kversusk2} is very similar to figure \ref{alg:Kversusk}. But there is here a definite decrease in computation time for SLM. Notice also that for larger $K$ the energy and error diverges for both SLM and Arnoldi. 


%\section{Integrating over looong time}%%%%%%%%%%%%%%%%%%%%%%%%%%%%%%%%%%%%%%%%%%%%%%%%%%%%%%%%%%%%%%%%%%%%%%%%%%%%%%%%
%!!!!!!!!!!!!!!!!!!!!!!!!Fjerne denne seksjonen??!!!!!!!!!!!!!!!!!!!!!!!!!!!!!!!!!!!!!\\
%This section will show how the energy changes when $T_s$ increases, with everything else constant.
%\subsection{Constant energy}
%\begin{figure}[H]
%        \centering
%        \begin{subfigure}[b]{0.3\textwidth}
%                \includegraphics[width=\textwidth]{../MATLAB/fig/longtime10.jpg}
%                \caption{ Trapezoidal rule without restart. Helpline increases with $T_s$ }
%                \label{fig:longtime10}
%        \end{subfigure}
%        ~
%        \begin{subfigure}[b]{0.3\textwidth}
%                \includegraphics[width=\textwidth]{../MATLAB/fig/longtime20.jpg}
%                \caption{ Forward Euler without restart. Helpline increases with $T_s^2$ }
%                \label{fig:longtime20}
%        \end{subfigure}
%        ~
%        \begin{subfigure}[b]{0.3\textwidth}
%                \includegraphics[width=\textwidth]{../MATLAB/fig/longtime30.jpg}
%                \caption{ Midpoint rule without restart. Helpline increases with $T_s$ }
%                \label{fig:longtime30}
%        \end{subfigure}
%        
%        \begin{subfigure}[b]{0.3\textwidth}
%                \includegraphics[width=\textwidth]{../MATLAB/fig/longtime11.jpg}
%                \caption{ Trapezoidal rule with restart. }
%                \label{fig:longtime11}
%        \end{subfigure}
%        ~
%        \begin{subfigure}[b]{0.3\textwidth}
%                \includegraphics[width=\textwidth]{../MATLAB/fig/longtime21.jpg}
%                \caption{ Forward Euler with restart. }
%                \label{fig:longtime21}
%        \end{subfigure}
%        ~
%        \begin{subfigure}[b]{0.3\textwidth}
%                \includegraphics[width=\textwidth]{../MATLAB/fig/longtime31.jpg}
%                \caption{ Midpoint rule with restart. }
%                \label{fig:longtime31}
%        \end{subfigure}
%        \caption{ The figures shows how the energy for the different methods change with the different integration methods, with and without restart. }
%        \label{fig:longtime}
%\end{figure}
%Without restart the energy for trapezoidal and midpoint rule increases linearly, while forward Euler's energy increases extremely fast.
%With restart the energy for all methods except DM increases very fast. The restart must be blamed for this, since it did not increase so fast when restart was not enabled. I would suspect that a larger increase in energy makes Arnoldi and SLM restart more, and more restarts add up to a larger energy, giving a feedback loop of diverging energy.
%\subsection{Varying energy}%%%%%%%%%%%%%%%%%%%%%%%%%%%%%%%%%%%%%%%%%%%%%%%%%%%%%%%%%%%%%%%%%%%%%%%%%%%%%%%%%%%%%%%%%%%
%\begin{figure}[H]
%        \centering
%        \begin{subfigure}[b]{0.3\textwidth}
%                \includegraphics[width=\textwidth]{../MATLAB/fig/longtime102.jpg}
%                \caption{ Trapezoidal rule without restart. Helpline increases with $T_s$ }
%                \label{fig:longtime102}
%        \end{subfigure}
%        ~
%        \begin{subfigure}[b]{0.3\textwidth}
%                \includegraphics[width=\textwidth]{../MATLAB/fig/longtime202.jpg}
%                \caption{ Forward Euler without restart. Helpline increases with $T_s^2$ }
%                \label{fig:longtime202}
%        \end{subfigure}
%        ~
%        \begin{subfigure}[b]{0.3\textwidth}
%                \includegraphics[width=\textwidth]{../MATLAB/fig/longtime302.jpg}
%                \caption{ Midpoint rule without restart. Helpline increases with $T_s$ }
%                \label{fig:longtime302}
%        \end{subfigure}
%        
%        \begin{subfigure}[b]{0.3\textwidth}
%                \includegraphics[width=\textwidth]{../MATLAB/fig/longtime112.jpg}
%                \caption{ Trapezoidal rule with restart. }
%                \label{fig:longtime112}
%        \end{subfigure}
%        ~
%        \begin{subfigure}[b]{0.3\textwidth}
%                \includegraphics[width=\textwidth]{../MATLAB/fig/longtime212.jpg}
%                \caption{ Forward Euler with restart. }
%                \label{fig:longtime212}
%        \end{subfigure}
%        ~
%        \begin{subfigure}[b]{0.3\textwidth}
%                \includegraphics[width=\textwidth]{../MATLAB/fig/longtime312.jpg}
%                \caption{ Midpoint rule with restart. }
%                \label{fig:longtime312}
%        \end{subfigure}
%        \caption{ The figures shows how the energy for the different methods change with the different integration methods, with and without restart.  }
%        \label{fig:longtime2}
%\end{figure}
%
%The difference between constant energy and varying energy is again not huge. The energy increases faster for trapezidal rule and midpoint rule in this case, but apart from that it seams to be the same.